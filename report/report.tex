\documentclass[a4paper,11pt]{article}
\usepackage[T1]{fontenc}
\usepackage[utf8]{inputenc}
\usepackage{lmodern}

\title{Stock Market Prediction}
\author{Mark Dunne}

\begin{document}

\maketitle
\tableofcontents

\begin{abstract}
\end{abstract}

\section{Introduction}
\subsection{Introduction}
\subsection{Project motivation}
\subsection{Importance and prevalence of the problem}
Explain size of the financial industry affected by the problem and why having better tools to predict the stock market may be important to the average person

\subsection{Project goals}

\subsection{Project outline}
How the project will meet those goals

\section{Background}
\subsection{The Stock Market}
\subsubsection{What is the stock market}
\subsubsection{How the stock market works}
\subsubsection{Components of the stock market}
\subsubsection{Terminology}

\subsection{Analysis of the problem}
\subsubsection{Explanation of the difficulty of the problem}
\subsubsection{Separation of profitability and accuracy}
\subsubsection{Temporal reach of prediction}
\subsubsection{Formal definition of the problem}

\subsection{Review of existing work}

\section{Methodology and Data}
\subsection{Tools Used}
\subsubsection{Python, Numpy, Pandas}
\subsubsection{Quantopian/Zipline and Pyalgotrade}
\subsubsection{Statsmodels}

\subsection{Data Used}
\subsubsection{Data sources}
\subsubsection{Format of the data}
\subsubsection{Adjusted prices}

\subsection{Simulation of strategies}
Similarity to real life

\subsection{Defining a successful model}
Statistical significance of a model

\section{Attacking the problem - Fundamental Analysis}

We begin by approaching the problem using Fundamental Analysis. Fundamental Analysis of stocks and shares is the earliest and simpliest form of prediction.

\subsection{PE Ratio}

\section{Attacking the problem - Technical Analysis}
\subsection{Hobbyist Approaches}
\subsection{Review of Metrics}
\subsection{OLMAR algorithm}
\subsection{StatsModels}

\section{Attacking the problem - Machine Learning}
\subsection{KNN on metrics}

\cite {website:pybrain-tutorial}

\bibliography{report}
\bibliographystyle{plain}

\end{document}
