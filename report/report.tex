\documentclass[a4paper,11pt]{article}
\usepackage[T1]{fontenc}
\usepackage[utf8]{inputenc}
\usepackage{lmodern}
\usepackage[numbers]{natbib}

\title{Stock Market Prediction}
\author{Mark Dunne}

\begin{document}

\maketitle
\tableofcontents

\begin{abstract}
\end{abstract}

\section{Introduction}
\subsection{Introduction}
\subsection{Project motivation}
\subsection{Importance and prevalence of the problem}
Explain size of the financial industry affected by the problem and why having better tools to predict the stock market may be important to the average person

\subsection{Project goals}

\subsection{Project outline}
How the project will meet those goals

\section{Background}
\subsection{The Stock Market}
\subsubsection{What is the stock market}
\subsubsection{How the stock market works}
\subsubsection{Components of the stock market}
\subsubsection{Terminology}
\subsubsection{The Efficient Market Hypothesis}

Throughout this area of investigation, there is a very large elephant in the room, the Efficient Market Hypothesis (EMH). The hypothesis talks specifically about an agents ability to profit from make inefficiencies, i.e when stocks and shares are mispriced by the market. Its strongest proponents would claim that the very title of this report, predicting stock market, is all but impossible. The EMH comes in three main forms.
\begin{itemize}
  \item The weak form of the efficient market hypothesis claims that prices fully
reflect the information implicit in the sequence of past prices. 

  \item The semi-strong form of the hypothesis asserts that prices reflect all relevant information that is publicly available
  
  \item the strong form of market efficiency asserts information that is known to any participant is reflected in market prices.
\end{itemize}
\cite{dimson1998brief}

Informally, the weak form implies that you cannot profit using strategies shaped on historic data, the semi-strong form implies that there is profit only to be made from insider trading, and the strong form says that even this is futile. Clearly if this project is to be of any success, we must hope that the hypothesis is wrong and does allow for a sufficiently intelligent agent to profit.

Luckily, many researchers do indeed question the validity of the hypothesis. There is evidence that the stock market does not always follow EMH. \citet{basu1983relationship} showed that certain analysis could yeild information useful in future market forecasts. This result questions the semi-strong and strong forms of the hypothesis, but does not necessarily break the weak form.

(TODO something that breaks the weak form)

\subsection{Analysis of the problem}
\subsubsection{Explanation of the difficulty of the problem}
\subsubsection{Separation of profitability and accuracy}
\subsubsection{Temporal reach of prediction}
\subsubsection{Formal definition of the problem}

\subsection{Review of existing work}

\section{Methodology and Data}
\subsection{Tools Used}
\subsubsection{Python, Numpy, Pandas}
\subsubsection{Quantopian/Zipline and Pyalgotrade}
\subsubsection{Statsmodels}

\subsection{Data Used}
\subsubsection{Data sources}
\subsubsection{Format of the data}
\subsubsection{Adjusted prices}

\subsection{Simulation of strategies}
Similarity to real life

\subsection{Defining a successful model}
Statistical significance of a model

\section{Attacking the problem - Fundamental Analysis}

We begin by approaching the problem using Fundamental Analysis. 

Fundamental Analysis of stocks and shares is one of the earliest and forms of market prediction. It takes the view that the market has mispriced an security, but over time the price will be corrected to its intrinsic value. If we can accurately calculate the intrinsic value of a security, e.g how much is one share of company \textit{X} actually worth, then we can choose to invest based on the difference between the current price and intrinsic value. 

\citet{graham1934security} laid the groundwork for the field with his book \textit{Security Analysis}.

\subsection{PE Ratio}

\section{Attacking the problem - Technical Analysis}
\subsection{Hobbyist Approaches}
\subsection{Review of Metrics}
\subsection{OLMAR algorithm}
\subsection{StatsModels}

\section{Attacking the problem - Machine Learning}
\subsection{KNN on metrics}

\cite {website:pybrain-tutorial}

\bibliography{report}
\bibliographystyle{plainnat}

\end{document}
